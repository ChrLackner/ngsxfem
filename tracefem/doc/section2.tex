\section{Discretization}
We want to solve 
\begin{subequations}
\label{formulation1}
\begin{align}
\partial_t u + \flow \cdot \nabla_\boundary u - \text{div}_\boundary(\alpha \nabla_\boundary u)=f \\
u(0,\cdot)=u_0
\end{align}
\end{subequations}
on $\boundary$, for $t\in(0,T)$ and a given $\flow$ and $u_0$. Let $n$ be to normal vector on $\boundary$ and  $P:=\Id-n^Tn$ be the orthogonal projection on the tangential space of $\boundary$. Then
\begin{align} 
\label{napG}
\nabla_\boundary u=P\nabla u
\end{align}
 and for $\domain \subset \R^3$
\begin{align}
\label{divG}
\text{div}_\boundary \flow = \sum_{k=1}^n \frac{\partial (P \flow)_k}{\partial x_k}
= \sum_{k=1}^n \frac{P \partial \flow_k}{\partial x_k}
\end{align}
We consider $\domain_1$ and $\domain_2$ on $\domain$, seperated  by $\boundary$ with a given triangulation $\mathcal{T}_h$. We use the standard finite element space of continuous piecewise linear functions $V_h$ on $\mathcal{T}_h$. To obtain a finite element space only on the boundary $\boundary$ we use the restriction of $V_h$ on $\boundary$, the trace finite element space $V^\boundary_h=\text{tr}|_\boundary V_h$.\\
To obtain $\boundary$ we use a given levelset function $\phi:\domain \to \R$, this means $\boundary$ is defined by a implizit manifold.\\
At first we want to solve the stationary case of (\ref{formulation1}), set $\partial_t u=0$.
\subsection{stationary case}
We want to solve 
\begin{align}
\label{probstat}
\flow \cdot \nabla_\boundary u + \text{div}_\boundary(\alpha \nabla_\boundary u)=f 
\end{align}
on $\boundary$. Multiplying be a testfunction $v\in V^\boundary_h$ and taking the integral we get
\begin{align*}
\int_\boundary  \flow \cdot \nabla_\boundary u v\ ds
- \int_\boundary \text{div}_\boundary(\alpha \nabla_\boundary u) v\ ds
= \int_\boundary f v\ ds
\end{align*}
using (\ref{napG}) and (\ref{divG}) we get 
\begin{align}
\int_\boundary  \flow \cdot P\nabla u v \ ds
- \int_\boundary \text{div}(\alpha P \nabla u) vds
= \int_\boundary f v ds\nonumber\\
\int_\boundary  \flow \cdot P\nabla u v\ ds
+ \int_\boundary \alpha (P \nabla u) \cdot \nabla v\ ds
= \int_\boundary f v  ds
\end{align}
The last step can done because of the rule of partial integration by Gau\ss \ with $\partial \boundary=\emptyset$. This gives us the weak formulation for (\ref{probstat}).\\
Find $u\in \Vx$ so that
\begin{align}
\label{weakstat}
\int_\boundary  (\flow \cdot P\nabla u) v\ ds
+ \int_\boundary \alpha (P \nabla u) \cdot \nabla v\ ds
= \int_\boundary f v\ ds \qquad \forall v \in \Vx\
\end{align}
Because of  $\Vx\subset H^1_0$ the equation is well defined.\\
We are on a finite functionspace in $\domain$, with base functions $(\phix)_{j=1}^N$. This means we can write $u=\sum_{k=1}^N u_k \phix_k$ and choose the coefficiant vector $u_k$ so that equation (\ref{weakstat}) is fullfilled. We can define $A \in \R^{N\times N}$ with entries
\begin{align}
\label{Aij}
A_{ij}:=\int_\boundary  (\flow \cdot P\nabla \phix_i) \phix_j\ ds + \int_\boundary \alpha (P \nabla \phix_i) \cdot \nabla \phix_j\ ds
\end{align}
and $b\in \R^N$ with $b_i=\int_\boundary f\phix_j \ ds$. We get $Au=b$ by choosing $v=\phix_j$ for $j=1,\dots,N$ in (\ref{weakstat}) and the weak solution $u$ of (\ref{probstat}) by solving
\begin{align}
u=A^{-1}b
\end{align}

\subsection{instationary case}
To solve the instationary case (\ref{formulation1}) we want to use the method of lines, solving a system of ODEs in time. Let $\Vx\subset H^1_0$ be the trace fem space in $\domain$ with the base functions $(\phix)_{j=1}^N$, using the weak formulation for the stationary part we are looking for a solution $u\in C^1([0,T],\Vx)$ which solves
\begin{align}
<\partial_t u,v> 
+\int_\boundary ( \flow \cdot P\nabla u) v\ ds
+ \int_\boundary \alpha (P \nabla u) \cdot \nabla v\ ds \nonumber
\\= \int_\boundary f v\ ds \qquad \forall v \in \Vx\
\end{align}
We define the matrices $M,A \in \R^{N\times N}$
\begin{align*}
M_{ij}:= \int_\boundary \phix_i \phix_j\ ds
\end{align*}
$A_{ij}$ as in (\ref{Aij})and $F(t)$ by
\begin{align*}
F(t)=\int_\boundary f(t) \phix_j\ ds
\end{align*}
We can write our solution $u\in C^1([0,T],\Vx)$ by base functions as $u(t)=\sum_{k=1}^N u_k(t) \phix_k$ with the coefficiant vector $\vec u$ to get the ODE in time
\begin{subequations}
\begin{align}
\label{MA}
M(\vec u)'(t)+A\vec u(t)=F(t)\\
u(0)=u_0
\end{align}
\end{subequations}
(\ref{MA}) gives us
\begin{align}
(\vec u)'(t)=M^{-1}F(t)-M^{-1}A\vec u(t)
\end{align}
we can approximate this ODE by using implizit euler. 
\begin{rmk}
The implizit euler method for $x'=f(x,t)$ for $x\in\R^N$, with stepsize $h=T/J$ a time discretization $t_k=hk$ and $x_k\approx x(t_k)$ for $k=0,\dots,J$ is given by
\begin{align*}
x_0=x(0)\\
x_{k+1}=x_k+h f(x_{k+1},t_{k+1})
\end{align*}
\end{rmk}
Using the implizit euler Method we define the initial value $\ucoef^0=\vec u_0$, using $\vec u_0$ as coefficiant vector of $u_0$ projected into $\Vx$ and for one time step we get
\begin{align*}
\ucoef^{k+1}&=\ucoef^k+h\left(M^{-1}F(t_{k+1})-M^{-1}A\ucoef^{k+1}\right)\\
\left(\Id + hM^{-1}A \right)\ucoef^{k+1}&=\ucoef^k+hM^{-1}F(t_{k+1})\\
\ucoef^{k+1}&=\left(\Id + hM^{-1}A \right)^{-1}(\ucoef^k+hM^{-1}F(t_{k+1}))\\
&=\left(h^{-1}M+A \right)^{-1}(h^{-1}M\ucoef^k+F(t_{k+1}))\\
&=\ucoef^k + \left(h^{-1}M+A \right)^{-1}(-A\ucoef^k+F(t_{k+1}))\\
\end{align*}